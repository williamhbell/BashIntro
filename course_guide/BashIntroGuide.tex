\documentclass[17pt,dvips]{foils}
\usepackage{graphics}
\usepackage{latexsym}
 %
 % Declare the title, author and date as you would in regular \LaTeX.
 %
\title{Basic Shell Programming}
 %
\author{W. H. Bell\\
       \texttt{W.Bell@cern.ch}
       }
 % This is optional
\date{\copyright 2010}

\MyLogo{\bash shell programming introduction}   % this is the default
\rightfooter{\textsf{\thepage}}   % this is the default \thepage
\leftheader{W. H. Bell}
\rightheader{2010}                % \filedate is defined above
 %
\newcommand\bs{\char '134}   %  a backslash character for the \tt font
 %
 % Now we can begin the document.  The first thing is the title page
 % on which we might put a VERY short abstract.
% Extra commands
\def\bash{\texttt{BASH}$\;$}

\begin{document}

\maketitle
 %
\begin{abstract}
\begin{center}
A basic overview of the BASH scripting language is given.  The
language is introduced using examples covering a few aspects at a
time.  Reference tables are provided for commonly used \bash
functionality.
\end{center}
\end{abstract}

%%%%%%%%%%%%%%%%%%%%%%%%%%%%%%%%%%%%%%%%%%%%%%%%%%%%%%%%%%%%%%%%%%%%%%%%%%%%

\foilhead{Introduction to \bash}
\noindent ``Bash is an sh-compatible command language interpreter that executes
commands read from the standard input or from a file.  Bash also
incorporates useful features from the Korn and C shells (ksh and csh).''

\begin{itemize}
\item Compound Commands
\begin{itemize}
\item Loops: for, while, until,
\item Conditional Statements: select, case, if 
\end{itemize}
\item File Operations
\begin{itemize}
\item Conditional Expressions
\item Reading and Writing to Files
\end{itemize}
\item Expansion and Its Uses
%Expansion is used when reading a file and in the pattern matching within the
%topic of String Operations.
\item Command line arguments
\item String Operations
\begin{itemize}
\item Conditional Expressions
\item Uses of Parameter Expansion
\item Other Commands
\end{itemize}
\item Functions
\item External Commands
\end{itemize}

\noindent Typing \texttt{man bash} gives a lot of extra information.


%%%%%%%%%%%%%%%%%%%%%%%%%%%%%%%%%%%%%%%%%%%%%%%%%%%%%%%%%%%%%%%%%%%%%%%%%%%%

\foilhead{Starting a Shell}
\begin{itemize}
\item Each user has a default shell: defined in passwd file.
\begin{verbatim}
]$ echo $SHELL
/bin/bash
\end{verbatim}
\item Can start another shell from the default shell either by typing
the name of the shell:
\begin{verbatim}
]$ bash
\end{verbatim}
or by creating a file with the \texttt{PATH} to the shell at the top
of the file, eg:
\begin{verbatim}
]$ vi example_01.bash
#!/bin/bash
~          
\end{verbatim} %$
\end{itemize}


%%%%%%%%%%%%%%%%%%%%%%%%%%%%%%%%%%%%%%%%%%%%%%%%%%%%%%%%%%%%%%%%%%%%%%%%%%%%

\foilhead{A Basic Shell Program}
\begin{itemize}
\item First create a file containing the script.
%
\begin{verbatim}
#!/bin/bash
echo "In the beginning..."
\end{verbatim}
%
\item The file does not need to have a specific file extension.  (For
the examples given in this course the \bash scripts have a suffix of
\texttt{.sh})
%
\item Then make the file executable
%
\begin{verbatim}
]$ chmod u+x ex1.sh
\end{verbatim}
%
\item And finally run the script.
%
\begin{verbatim}
]$ ./ex1.sh
In the beginning...
\end{verbatim}
%$
\end{itemize}

\vspace{1cm}

%%%%%%%%%%%%%%%%%%%%%%%%%%%%%%%%%%%%%%%%%%%%%%%%%%%%%%%%%%%%%%%%%%%%%%%%%%%%

\foilhead{White Spaces and New Lines}
\begin{itemize}
\item Spaces and new lines are very important in shell programming.
\item In many languages white spaces are ignored i.e. the compiler or
interpreter skips over them.
\item This is not so in basic shell programming.  Try taking some of
the white spaces out of the following examples and see the result.
\end{itemize}

%%%%%%%%%%%%%%%%%%%%%%%%%%%%%%%%%%%%%%%%%%%%%%%%%%%%%%%%%%%%%%%%%%%%%%%%%%%%

\foilhead{Compound Commands: Loops: \texttt{for}}
\begin{verbatim}
#!/bin/bash

word="a b c"
i=0
 
# Read each character from $word and and assign
# it to $name
for name in $word; do
 
  # Use 'let' to increment i.
  let i++
 
  # Print the value of $name
  echo $name
 
done
 
echo "Looped $i times."
\end{verbatim}
{\bf example\_02.sh}: A program to demonstrate a type 1 \texttt{for} loop.
%
% Notice the variable word is given a value without using the '$'
% prefix and then the value is used with a dollar prefix.
% Spaces here are very important: adding a space between i and = will
% cause a compilation error.
% Let provides arithmetic operations, which are not allowed outside 
% (( )) brackets.
% Finally 'echo' is a command normally accessible to users.  Other
% commands can be used in a simular fashion.

%%%%%%%%%%%%%%%%%%%%%%%%%%%%%%%%%%%%%%%%%%%%%%%%%%%%%%%%%%%%%%%%%%%%%%%%%%%%

\foilhead{Compound Commands: Loops: \texttt{for}}
\begin{verbatim}
#!/bin/bash
 
# Loop from 0 to 9.
for((i=0; i<10; i++)) ; do
 
  # Append the string form of i to the
  # end of the string j
  j=$j$i
 
done
 
echo $j
\end{verbatim}
%$
{\bf example\_03.sh}: A program to demonstrate a type 2 \texttt{for} loop.
% (( )) brackets contain arithmetic operations.

%%%%%%%%%%%%%%%%%%%%%%%%%%%%%%%%%%%%%%%%%%%%%%%%%%%%%%%%%%%%%%%%%%%%%%%%%%%%

\foilhead{Compound Commands: Loops: \texttt{while}, \texttt{until}}
\begin{verbatim}
#!/bin/bash

nloops=3
i=0
 
echo "while loop"
while [[ $i<$nloops ]]; do
    echo $i
    let i++
done
 
echo
echo "until loop"
i=0
until [[ $i>$nloops ]]; do
    echo $i
    let i++
done
\end{verbatim}
%$
{\bf example\_04.sh}: A program to demonstrate \texttt{while} and 
\texttt{until} loops.
% The until loop version loops one more time wrt the while loop form.
% [[ ]] contains boolean expression.  Must have a space between while
% and [[

%%%%%%%%%%%%%%%%%%%%%%%%%%%%%%%%%%%%%%%%%%%%%%%%%%%%%%%%%%%%%%%%%%%%%%%%%%%%

\foilhead{Compound Commands: \\Conditional Statements: \texttt{if}}
\begin{verbatim}
#!/bin/bash
 
for ((i=0;i<3;i++)) do
    if [[ $i == 1 ]]; then
        echo "Turnip"
    elif [[ $i == 2 ]]; then
        echo "Potato"
    else
        echo "Carrot"
    fi
done
\end{verbatim}
%
{\bf example\_05.sh}: A program to demonstrate \texttt{if}, \texttt{elif}, 
\texttt{else} conditional statements.
%

%%%%%%%%%%%%%%%%%%%%%%%%%%%%%%%%%%%%%%%%%%%%%%%%%%%%%%%%%%%%%%%%%%%%%%%%%%%%

\foilhead{File Operations: Conditional Expressions}

\begin{verbatim}
#!/bin/bash
 
files="test test_dir test_link"
 
for file in $files; do
  if [[ -a $file ]]; then
    echo "File $file Exists"
  fi
   
  if [[ -f $file ]]; then
    echo "File $file is a regular file"
  fi
   
  if [[ -d $file ]]; then
    echo "File $file is a directory"
  fi
                                                                                
  if [[ -h $file ]]; then
    echo "File $file is a symbolic link"
  fi
done
\end{verbatim}
%$
{\bf example\_06.sh}: A program to that uses conditional expressions to test for
the presence of a file.

\foilhead{File Operations: Conditional Expressions}
\begin{itemize}
\item Before running example 6.
\begin{verbatim}
]$ touch test; mkdir test_dir
]$ ln -s test test_link
\end{verbatim}
%$
%\item Then try removing the file, directory etc.  Try replacing
%\texttt{\$files} with \texttt{`ls`}.
\end{itemize}
{\tiny
\begin{tabular}{|l|l|} \hline \hline
Usage & Result \\ \hline \hline
-a {\em file} & True if {\em file} exists\\ \hline
-b {\em file} & True if {\em file} exists and is a block special file\\ \hline
-c {\em file} & True if {\em file} exists and is a character special file\\ \hline
-d {\em file} & True if {\em file} exists and is a directory. \\ \hline
-e {\em file} & True if {\em file} exists\\ \hline
-f {\em file} & True if {\em file} exists and is a regular {\em file}\\ \hline
-g {\em file} & True if {\em file} exists and is set-group-id\\ \hline
-h {\em file} & True if {\em file} exists and is a symbolic link\\ \hline
-k {\em file} & True if {\em file} exists and its ``sticky'' bit is set\\ \hline
-p {\em file} & True if {\em file} exists and is a named pipe (FIFO)\\ \hline
-r {\em file} & True if {\em file} exists and is readable\\ \hline
-s {\em file} & True if {\em file} exists and has a size greater than zero\\ \hline
-u {\em file} & True if {\em file} exists and its set-user-id bit is set\\ \hline
-w {\em file} & True if {\em file} exists and is writable\\ \hline
-x {\em file} & True if {\em file} exists and is executable\\ \hline
-O {\em file} & True if {\em file} exists and is owned by the effective user id\\ \hline
-G {\em file} & True if {\em file} exists and is owned by  the  effective group id\\ \hline
-L {\em file} & True if {\em file} exists and is a symbolic link\\ \hline
-S {\em file} & True if {\em file} exists and is a socket\\ \hline
-N {\em file} & True  if {\em file} exists and has been modified since it was last read\\ \hline
{\em file1} -nt {\em file2}  & True if {\em file1} is newer than {\em file2}\\ \hline
{\em file1} -ot {\em file2}  & True if {\em file1} is older than {\em file2}\\ \hline \hline
\end{tabular}
}
% nt and ot are based on the modification date.
% Missing off the -ef same inode and device.

%%%%%%%%%%%%%%%%%%%%%%%%%%%%%%%%%%%%%%%%%%%%%%%%%%%%%%%%%%%%%%%%%%%%%%%%%%%%

\foilhead{File Operations: Reading and Writing Files}
\begin{verbatim}
#!/bin/bash

words="electron muon tau"
outputfile="test.out"
 
rm -f $outputfile
 
for name in $words; do
  echo $name >> $outputfile
done
 
echo "cat $outputfile:"
cat $outputfile
 
echo
echo "Reading the words back in."
for name in $(<$outputfile); do
  str="$str$name, "
done
echo $str
\end{verbatim}
%$
{\bf example\_07.sh}: A program demonstrate file i/o using output redirection.

% File i/o is possible with blobs (pointer like type.)

%%%%%%%%%%%%%%%%%%%%%%%%%%%%%%%%%%%%%%%%%%%%%%%%%%%%%%%%%%%%%%%%%%%%%%%%%%%%

\foilhead{Expansion and Its Uses}
\begin{verbatim}
#!/bin/bash
 
echo "Brace expansion - 1{2,3,4}5:"
echo 1{2,3,4}5
 
echo "Tilde Expansion HOME - ~:"
echo ~
 
echo "Tilde Expansion PWD - ~+:"
echo ~+
 
echo "Tilde Expansion wbell's HOME:"
echo ~wbell/
\end{verbatim}
{\bf example\_08.sh}: A program to demonstrate brace and tilde expansion.

%%%%%%%%%%%%%%%%%%%%%%%%%%%%%%%%%%%%%%%%%%%%%%%%%%%%%%%%%%%%%%%%%%%%%%%%%%%%

\foilhead{Expansion and Its Uses}
\begin{verbatim}
#!/bin/bash
  
i=1
j=4

i=$((++i*j))
                                                                         
echo $i
\end{verbatim}
{\bf example\_09.sh}: A program to demonstrate arithmetic expansion

\begin{itemize}
\item Provides functionality of \texttt{let} i.e. \texttt{+}, 
\texttt{-}, \texttt{*}, \texttt{/}, \texttt{**}
\item Anything more complicated i.e. sine, sqrt etc: use \texttt{bc} or 
\texttt{perl}
\end{itemize}

%%%%%%%%%%%%%%%%%%%%%%%%%%%%%%%%%%%%%%%%%%%%%%%%%%%%%%%%%%%%%%%%%%%%%%%%%%%%

\foilhead{Command Line Arguments}
\begin{verbatim}
#!/bin/bash

echo "The number of args following the command = $#";

for arg in $* ; do
    str="$str $arg"
done

echo "$0$str"
\end{verbatim}
{\bf example\_10.sh}: A simple program to illustrating how command line
arguments can be read inside a shell program.
% Change the ex number when the others above are known.

\begin{itemize}
\item Commands can be accessed directly by using \texttt{\$}$n$ where
$n$ is the number of the command. E.g.
\begin{verbatim}
echo "The first argument is $1"
\end{verbatim}
%$
\item Be careful to check the value is defined before using it.
\end{itemize} 

%%%%%%%%%%%%%%%%%%%%%%%%%%%%%%%%%%%%%%%%%%%%%%%%%%%%%%%%%%%%%%%%%%%%%%%%%%%%


\foilhead{String Operations: Conditional Expressions}
\begin{verbatim}
#!/bin/bash
 
if [[ -n $CHECK_ME ]]; then
  echo "CHECK_ME = $CHECK_ME"
else
  echo "CHECK_ME is unset."
fi
\end{verbatim}
{\bf example\_11.sh}: A program to demonstrate the use of
conditional string operators.

\noindent To test the example program try setting and unsetting the CHECK\_ME
environmental variable:
\begin{verbatim}
]$ export CHECK_ME=1
]$ unset CHECK_ME
\end{verbatim}
Run the script before and after the environmental variable is set.

\foilhead{String Operations: Conditional Expressions}
\begin{center}

\begin{tabular}{|l|l|} \hline \hline
Usage & Result \\ \hline \hline
-z {\em string} & True if the length of string is zero \\ \hline
-n {\em string} & True if the length of string is non-zero \\ \hline
{\em string1} == {\em string2} & True if the strings are equal \\ \hline
{\em string1} != {\em string2} & True if the strings are not equal \\ \hline \hline
\end{tabular}

\noindent A summary of the most useful conditional string operators.
\end{center}
% Have missed out the strange local dependent string > and <.

%%%%%%%%%%%%%%%%%%%%%%%%%%%%%%%%%%%%%%%%%%%%%%%%%%%%%%%%%%%%%%%%%%%%%%%%%%%%
\foilhead{String Operations: Parameter Expansion}
\begin{verbatim}
#!/bin/bash

somestring=abcdef

echo "length = ${#somestring}"
   
i=2
echo "After $i characters ${somestring:$i}"   
echo "Before $i characters ${somestring: -$i}"
   
j=2
echo "From char $i to of length $j ${somestring:$i:$j}"
\end{verbatim}
{\bf example\_12.sh}: A script demonstrating substring selection via Parameter Expansion.

%%%%%%%%%%%%%%%%%%%%%%%%%%%%%%%%%%%%%%%%%%%%%%%%%%%%%%%%%%%%%%%%%%%%%%%%%%%%

\foilhead{String Operations: Parameter Expansion}
\begin{verbatim}
#!/bin/bash

parameter="filename.dat"
word=".dat"

remainder=${parameter%$word}

echo "parameter=$parameter word=$word"
echo "remainder=$remainder"
\end{verbatim}
%$
{\bf example\_13.sh}: A script to remove part of a string using Parameter Expansion.

\begin{itemize}
\item There are two types of this sort of parameter expansion.
\begin{itemize}
  \item \texttt{\$\{parameter\#word\}} - Matching the beginning.
  \item \texttt{\$\{parameter\%word\}} - Matching the end.
\end{itemize}
\item One \# or \% character for the shortest and two for the longest matching case.
\end{itemize}

%%%%%%%%%%%%%%%%%%%%%%%%%%%%%%%%%%%%%%%%%%%%%%%%%%%%%%%%%%%%%%%%%%%%%%%%%%%%

\foilhead{String Operations: Parameter Expansion}
\begin{verbatim}
#!/bin/bash

parameter="filename.dat"
pattern=".dat"
string=".root"

new_filename=${parameter/$pattern/$string}

echo "parameter=$parameter pattern=$pattern"
echo "string=$string"
echo "new_filename=$new_filename"
\end{verbatim}
%$
{\bf example\_14.sh}: A script to demonstrate string substitution using Parameter Expansion.

\begin{itemize}
\item The \texttt{pattern} is a pattern and not a \texttt{word}.
\end{itemize}

%%%%%%%%%%%%%%%%%%%%%%%%%%%%%%%%%%%%%%%%%%%%%%%%%%%%%%%%%%%%%%%%%%%%%%%%%%%%

\foilhead{String Operations: Pattern Matching}
\begin{verbatim}
#!/bin/bash

filename1="string"
filename2=" string "
match1=$filename1
match2=" string"

if [[ "$filename1" == "$match1" ]]; then
  echo "\"$match1\" matches \"$filename1\""
fi
 
if [[ "$filename2" == *"$match1"* ]]; then
  echo "\"$match1\" matches \"$filename2\""
fi

if [[ "$filename2" == "$match2"* ]]; then
  echo "\"$match2\" matches \"$filename2\""
fi
\end{verbatim}
%$
{\bf example\_15.sh}: A script demonstrating string pattern matching.


%%%%%%%%%%%%%%%%%%%%%%%%%%%%%%%%%%%%%%%%%%%%%%%%%%%%%%%%%%%%%%%%%%%%%%%%%%%%

\foilhead{String Operations: Other Commands}
A range of commands outside of the bash language can be used to operate on 
strings.
\begin{itemize}
\item \texttt{expr} - Provides many string operations together with logic and numeric functions.  (Type \texttt{info expr} for more information.) 
\item \texttt{sed} -  A stream editor used to perform operations on text.
\item \texttt{awk} - Is the interpreter for The AWK Programming Language.
\end{itemize} 

%%%%%%%%%%%%%%%%%%%%%%%%%%%%%%%%%%%%%%%%%%%%%%%%%%%%%%%%%%%%%%%%%%%%%%%%%%%%

\foilhead{Functions}
\begin{verbatim}
...
usage() {
  echo ""
  echo " Usage: $0 <directory>"
  echo ""
  exit 1
}

baddir() {
  echo ""
  echo " $1 can not be listed"
  echo ""
}
...
\end{verbatim}
%$
{\bf An extract from example\_16.sh}: Two functions: one using a global parameter, the other a local one.

%%%%%%%%%%%%%%%%%%%%%%%%%%%%%%%%%%%%%%%%%%%%%%%%%%%%%%%%%%%%%%%%%%%%%%%%%%%%

\foilhead{Functions}
\begin{verbatim}
...
# Check at least one argument is given
if [[ -z $1 ]]; then
  usage
fi
...
else
  baddir $dir
fi
...
\end{verbatim}
{\bf An extract from example\_16.sh}: Calling the two functions previously defined.

%%%%%%%%%%%%%%%%%%%%%%%%%%%%%%%%%%%%%%%%%%%%%%%%%%%%%%%%%%%%%%%%%%%%%%%%%%%%

\foilhead{External Commands}
\begin{verbatim}
...
files=$(ls $dir)
if [[ $? == 0 ]]; then
...
\end{verbatim}
%$
{\bf An extract from example\_16.sh}: Demonstrating how to execute external commands.
\begin{itemize}
\item \texttt{\$?} Contains the return value from the command.
\item The return statement is the last return value.  Therefore do not put an \texttt{if} statement between the command and the test on \texttt{\$?} 
\end{itemize}

\end{document}
\endinput
